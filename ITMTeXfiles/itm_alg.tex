
   \magnification=\magstep1 
   \newtoks\title  \newtoks\authors \newtoks\authorattr 
   \title={The ITS Irregular Terrain Model, version 1.2.2\cr
       The Algorithm}
   \authors={George Hufford}
     \authorattr={The author is with the Institute for Telecommunication
     Sciences, National Telecommunications and Information Administration,
       U.S.~Department of Commerce, Boulder, Colorado\ \ 80303.}

   \def\footfont{\font\ftq=cmr8 \ftq \baselineskip=9pt 
           \lineskip=0pt 
           \def\textindent##1{\noindent##1\enspace\ignorespaces}} 
   \parindent=0pt 
   \parskip=10pt plus 2pt minus 1pt 


   \def\us#1{\ {\rm#1}}
   \font\Csc=cmcsc10 \def\Fortran{{\Csc Fortran}}
   \def\sqrtt#1{{\textstyle\sqrt{#1}}}
     
   \def\newsec#1\par{\goodbreak\bigskip\message{#1}\leftline{\bf#1}
      \nobreak\smallskip\vskip-\parskip}
   \def\inplist#1=#2{\hbox to \hsize{
       \hbox to 1.5truein{\hfil #1}\hskip .3truein
       \vtop{\hsize=4truein\noindent\hangindent=2em\strut#2\strut\par}
       \hfil}\smallskip}

     {\footline={\hfil}\pageno=-1 
         \hrule height0pt \vskip 3.75truein 
         \tabskip=0pt plus 1fil 
         \font\ftq=cmss10 scaled \magstep2\ftq \openup3pt 
         \halign to \hsize{\hfil#\hfil\cr\the\title\cr} 
         \bigskip 
         \rm\halign to \hsize{\hfil#\hfil\cr\the\authors\cr} 
          \vfill 
           \centerline{National Telecommunications and Information 
             Administration} 
           \centerline{Institute for Telecommunication Sciences} 
           \centerline{325 Broadway} 
           \centerline{Boulder, CO 80303-3328, U.S.A.}       \eject}
         \hrule height0pt\vfil\eject
    
    \footline={\ifnum\pageno=1 \hfil \else 
           \hfil\tenrm\folio\hfil \fi} \pageno=1 

         \begingroup 
         \font\ftq=cmb10 scaled \magstep1 
         \tabskip=0pt plus 1fil 
         {\ftq\openup3pt\halign to \hsize{\hfil#\hfil\cr\the\title\cr}} 
         \bigskip 
         \rm\halign to
   \hsize{\hfil#\hfil\cr\the\authors{\footfont\footnote*%
   {\the\authorattr}}\cr} 
     \vskip 30truept plus 10truept minus 5truept  \medskip \endgroup
   In this report I have attempted to write down the algorithm of the ITM
   (the Irregular Terrain Model) as it is currently implemented.  There are
   probably, however, some features of the model that are not documented
   here.  I hope they are not many, and that one need not refer too often
   to the original \Fortran\ source code.

   The purpose of the model is to estimate some of the characteristics of a
   received signal level for a radio link.  This usually means cumulative
   distributions for what really appears to be a random phenomenon.

     The original model was developed in the late 1960's when land mobile
   radio and television broadcasting were important systems that required
   better engineering.  Perhaps that explains the emphasis on low and
   hidden antennas and on the long-distance fields that might cause
   interference.

   \newsec Historical Bibliography.

   The reports listed here are probably the more important mileposts in the
   development of the ITM.  They are not necessarily relevant to the
   present algorithm, but are included here for their historical interest.

   \begingroup 
      \parindent=20pt \parskip=6pt plus2pt 
      \def\[#1]{\indent\llap{[#1]\enspace}\ignorespaces} 
      \frenchspacing \everypar={\hangindent=\parindent}
   Barsis, A. P., and P. L. Rice (1963), Prediction and measurement of VHF
   field strength over irregular terrain using low antenna heights, NBS
   Report 8891.

   Rice, P. L., A. G. Longley, K. A. Norton, and A. P. Barsis (1967),
   Transmission loss predictions for tropospheric communication circuits,
   NBS Tech. Note 101, vols I and II.  Available from NTIS, Access. Nos.
   AD-687-820 and AD-687-821.

   Longley, A. G., and P. L. Rice (1968), Prediction of tropospheric radio
   transmission loss over irregular terrain---a computer method 1968, ESSA
   Tech. Report ERL65-ITS67.  Available from NTIS, Access. No. AD-676-874.

   Longley, A. G., and R. K. Reasoner (1970), Comparison of propagation
   measurements with predicted values in the 20 to 10,000~MHz range, ESSA
   Tech. Report ERL148-ITS97.  Available from NTIS, Access. No. ??

   Longley, A. G. (1978), Radio propagation in urban areas, OT Report
   78-144.  Available from NTIS, Access. No. PB-281-932.

   Longley, A. G., and G. A. Hufford (1979), Memorandum to users of ARPROP,
   of its subroutine ACR, or of the Longley-Rice model in its latest
   version (version 1.2); subject: Modifications.  Unpublished.

   Hufford, G. A., A. G. Longley, and W. A. Kissick (1982), A guide to the
   use of the ITS Irregular Terrain Model in the area prediction mode, NTIA
   Report 82-100.  Availbale from NTIS, Access. No. PB82-217977.

   Hufford, G. A. (1985), Memorandum to users of the ITS Irregular Terrain
   Model; subject: Modifications.  Unpublished.

   \endgroup
   \newsec 1. Input.

   We picture a radio link located in some region on the earth.  Then the
   input needed by the model is a proper description of that link.
   The two modes of the model---the {\it area prediction mode} and the {\it
   point-to-point mode}---are distinguished mostly by the amount of input
   data required.  The point-to-point mode must provide details of the
   terrain profile of the link that the area prediction mode will estimate
   using empirical medians.  Since in other respects the two modes follow
   very similar paths we shall try here to treat both in parallel. 

   The two terminals of the link we denote as terminals 1 and 2, leaving it
   to the user to identify which is transmitter and which receiver.

    We try always to use consistent units and we think the user can modify
   our statements to fit any desired basic units.  For ourselves, we prefer
   SI units and express lengths (and distances) in meters. Exceptions to
   the rule of consistent units might include our measure of atmospheric
   refractivity (N-units or parts per million) and our measure of losses,
   attenuations, gains, etc. (in decibels).

   \newsec 1.1. General input for both modes of usage.

   \medskip
   \inplist $d$={Distance between the two terminals.}
   \inplist $h_{g1}$, \ $h_{g2}$={Antenna structural heights.}
   \inplist $k$={Wave number, measured in units of reciprocal lengths;
   see Note~1.}
   \inplist $\Delta h$={Terrain irregularity parameter.}
   \inplist $N_s$={Minimum monthly mean surface refractivity, measured in
   N-units; see Note~2.}
   \inplist $\gamma_e$={The earth's effective curvature, measured in
   units of reciprocal length; see Note~3.}
   \inplist $Z_g$={Surface transfer impedance of the ground---a complex,
   dimensionless number; see Note~4.}
   \inplist radio climate={Expressed qualitatively as one of a number of
   discrete climate types.}
   Note 1.  The wave number is that of the carrier or central frequency.
   It is defined to be
   $$ k=2\pi/\lambda=f/f_0 \qquad\hbox{with $f_0=47.70\us{MHz\cdot m}$}
                \eqno(1.1)$$
   where $\lambda$ is the wave length, $f$ the frequency.  (Here and
   elsewhere we have assumed the speed of light in air is
   299.7~m/$\mu$s.)

   Note 2.  To simplify its representation, the surface refractivity is
   sometimes given in terms of $N_0$, the surface refractivity ``reduced
   to sea level.''  When this is the situation, one must know the general
   elevation $z_s$ of the region involved, and then
   $$ N_s=N_0 e^{-z_s/z_1}\qquad\hbox{with $z_1=9.46\us{km}$}.\eqno(1.2)$$

   Note 3.  The earth's effective curvature is the reciprocal of the
   earth's effective radius and may be expressed as
   $$ \gamma_e=\gamma_a/K $$
   where $\gamma_a$ is the earth's actual curvature and $K$ is the
   ``effective earth radius factor.''  The value is normally determined
   from the surface refractivity using the empirical formula
   $$ \gamma_e=\gamma_a(1-0.04665\,e^{N_s/N_1})\eqno(1.3)$$
   where
   $$N_1=179.3\hbox{ N-units, and}\quad\gamma_a=157\cdot10^{-9}
       \us m^{-1}=157\hbox{ N-units/km}.$$

   Note 4.  The ``surface transfer impedance'' is normally defined in
   terms of the relative permittivity $\epsilon_r$ and conductivity
   $\sigma$ of the ground, and the polarization of the radio waves
   involved.  In these terms, we have
   $$ Z_g=\cases{
            \sqrt{\epsilon_r^\prime-1}&horizontal polarization\cr
            \sqrt{\epsilon_r^\prime-1}/\epsilon_r^\prime&vertical
                      polarization\cr}  \eqno(1.4)$$
   where $\epsilon_r^\prime$ is the ``complex relative permittivity''
   defined by
   $$\epsilon_r^\prime=\epsilon_r+iZ_0\sigma/k,\qquad
                        Z_0=376.62\us{ohm}.\eqno(1.5)$$
   The conductivity $\sigma$ is normally expressed in siemens (reciprocal
   ohms) per meter.
   \newsec 1.2. Additional input for the area prediction mode.

   \medskip
   \inplist siting criteria={Criteria describing the care taken at each
   terminal to assure good radio propagation conditions.  This is
   expressed qualitatively in three steps: at random, with care, and with
   great care.}

   \newsec 1.3. Additional input for the point-to-point mode.

   \medskip
   \inplist$h_{e1}$, \ $h_{e2}$={Antenna effective heights.}
   \inplist$d_{L1}$, \ $d_{L2}$={Distances from each terminal to its
   radio horizon.}
   \inplist$\theta_{e1}$, \ $\theta_{e2}$={Elevation angles of the
   horizons from each terminal at the height of the antennas.  These are
   measured in radians.}
   These quantities, together with $\Delta h$, are all geometric and
   should be determined from the terrain profile that lies between the
   two terminals.  We shall not go into detail here.

   The ``effective height'' of an antenna is its height above an
   ``effective reflecting plane'' or above the ``intermediate
   foreground'' between the antenna and its horizon.  A difficulty with
   the model is that there is no explicit definition of this quantity,
   and the accuracy of the model sometimes depends on the skill of the
   user in estimating values for these effective heights.


   In the case of a line-of-sight path there are no horizons, but the
   model still requires values for $d_{Lj}$, \ $\theta_{ej}$, \ $j=1, 2$.
   They should be determined from the formulas used in the area
   prediction mode  and listed in Section~3 below.  Now it may happen
   that after these computations one discovers
   $d>d_L=d_{L1}+d_{L2}$,
   implying that the path is a beyond-horizon one.  Noting that $d_L$ is
   a monotone increasing function of the $h_{ej}$ we can assume these
   latter have been underestimated and that they should be increased by a
   common factor until $d_L=d$.
   \newsec 2. Output.

   The output from the model may take on one of several forms at the
   user's option.  Simplest of these forms is just the {\it reference
   attenuation} $A_{\rm ref}$.  This is the {\it median} attenuation
   relative to a free space signal that should be observed on the set of
   all similar paths during times when the atmospheric conditions
   correspond to a standard, well-mixed, atmosphere.

   The second form of output provides the two- or three-dimensional
   cumulative distribution of attenuation in which time, location, and
   situation variability are all accounted for.  This is done by giving
   the {\it quantile} $A(q_T,q_L,q_S)$, the attenuation that will not be
   exceeded as a function of the fractions of time, locations, and
   situations.  One says
   {\narrower\narrower\noindent\sl In $q_S$ of the situations there will
   be {\rm at least} $q_L$ of the locations where the attenuation {\rm
   does not exceed} $A(q_T,q_L,q_S)$ for {\rm at least} $q_T$ of the
   time.}

   When the point-to-point mode is used on particular, well-defined paths
   with definitely fixed terminals, there is no location variability, and
   one must use a two-dimensional description of cumulative
   distributions.  One can now say {\sl With probability (or confidence)
   $q_S$ the attenuation will not exceed $A(q_T,q_S)$ for at least $q_T$
   of the time}.  The same effect can be achieved by setting $q_L=0.5$ in
   the three-dimensional formulation.

   On some occassions it will be desireable to go beyond the
   three-dimensional quantiles and to treat directly the underlying model
   of variability.  For example, consider the case of a communications link
   that is to be used once and once only.  For such a ``one-shot'' system
   one is interested only in what probability or confidence an adequate
   signal is received that once.  The three-dimensional distributions used
   above must now be combined into one.
   \newsec 3. Preparatory Calculations.

   We start with some preliminary calculations of a geometric nature.

   \newsec 3.1. Preparatory calculations for the area prediction mode.

   The parameters $h_{ej}$, \ $d_{Lj}$, \ $\theta_{ej}$, \ $j=1,2$, which
   are part of the input in the point-to-point mode are, in the area
   prediction mode, estimated using empirical formulas in which $\Delta
   h$ plays an important role.

   First, consider the effective heights.  This is where the siting
   criteria are used.  We have
   $$h_{ej}=h_{gj}\qquad\hbox{if terminal $j$ is sited at random}.
          \eqno(3.1)$$
   Otherwise, let
   $$B_j=\cases{5\ {\rm m}&if terminal $j$ is sited with care\cr
               10\ {\rm m}&if terminal $j$ is sited with great
   care.\cr}$$
   Then
   $$B_j'=(B_j-H_1)\sin \bigl({\pi \over 2} \min(h_{g1}/H_2,1)\bigr)
           +H_1 \qquad\hbox{with $H_1=1\us m,\ H_2=5\us m$},$$
   and
   $$h_{ej}=h_{gj}+B_j'e^{-2h_{gj}/\Delta h}.\eqno(3.2)$$

   The remaining parameters are quickly determined.
   $$ d_{Lsj}=\sqrtt{2h_{ej}/\gamma_e}$$
   $$ d_{Lj}=d_{Lsj}\exp\bigl[-0.07\sqrtt{
       \Delta h/\max(h_{ej},H_3)}\,\bigr]
       \qquad\hbox{with $H_3=5\us m$},\eqno(3.3)$$
    and finally,
   $$ \theta_{ej}=[0.65\,\Delta h(d_{Lsj}/d_{Lj}-1)-2h_{ej}]/d_{Lsj}.
       \eqno(3.4)$$

   \newsec 3.2. Preparatory calculations for both modes.

   \medskip
   $$\eqalignno{d_{Lsj}&=\sqrtt{2h_{ej}/\gamma_e},\qquad j=1,2&(3.5)\cr
        d_{Ls}&=d_{Ls1}+d_{Ls2}&(3.6)\cr
        d_L&=d_{L1}+d_{L2}&(3.7)\cr
      \theta_e&=\max(\theta_{e1}+\theta_{e2},-d_L\gamma_e).&(3.8)\cr}$$

   We also note here the definitions of two functions of a distance s:
   $$\Delta h(s)=(1-0.8\,e^{-s/D})\Delta h\qquad\hbox{with
         $D=50\us{km}$},\eqno(3.9)$$and
   $$\sigma_h(s)=0.78\,\Delta h(s)\exp\bigl[-(\Delta h(s)/H)^{1/4}\bigr]
      \qquad\hbox{with $H=16\us m$}.\eqno(3.10)$$
   \newsec 4. The Reference Attenuation.

   The reference attenuation is determined as a function of the distance
   $d$ from the piecewise formula
   $$A_{\rm ref}=\cases{\max\bigl(0,A_{el}+K_1d+K_2\ln(d/d_{Ls})\bigr)
                         &$d \le d_{Ls}$\cr
                    A_{ed}+m_dd&$d_{Ls}\le d \le d_x$ \cr
                    A_{es}+m_sd&$d_x \le d$ \cr}\eqno(4.1)$$
   where the coefficients $A_{el}$, $K_1$, $K_2$, $A_{ed}$, $m_d$,
   $A_{es}$, $m_s$, and the distance $d_x$ are calculated using the
   algorithms below.  The three intervals defined here are called the
   line-of-sight, diffraction, and scatter regions, respectively.  The
   function in (4.1) is continuous so that at the two endpoints where
   $d=d_{Ls}$ or $d_x$ the two formulas give the same results.  It
   follows that instead of seven independent coefficients there are
   really only five.

   \newsec 4.1. Coefficients for the diffraction range.

   Set
   $$\eqalignno{X_{ae}&=(k\gamma_e^2)^{-1/3}&(4.2)\cr
        d_3&=\max(d_{Ls},d_L+1.3787\,X_{ae})&(4.3)\cr
        d_4&=d_3+2.7574\,X_{ae}&(4.4)\cr
        A_3&=A_{\rm diff}(d_3)&(4.5)\cr
        A_4&=A_{\rm diff}(d_4)&(4.6)\cr}$$
   where $A_{\rm diff}$ is the function defined below.  The formula for
   $A_{\rm ref}$ in the diffraction range is then just the linear
   function having the values $A_3$ and $A_4$ at the distances $d_3$ and
   $d_4$, respectively.  Thus
   $$\eqalignno{m_d&=(A_4-A_3)/(d_4-d_3)&(4.7)\cr
             A_{ed}&=A_3-m_dd_3.&(4.8)\cr}$$

   \newsec 4.1.1. The function $A_{\rm diff}(s)$.

   We first define the weighting factor
   $$w={1 \over 1+0.1\,\sqrtt Q}\eqno(4.9)$$
   with
   $$Q=\min\bigl({k \over 2\pi}\Delta h(s),1000)\left(
      {h_{e1}h_{e2}+C \over h_{g1}h_{g2}+C}\right)^{1/2}+
      {d_L+\theta_e/\gamma_e \over s}$$
   and
   $$C=\cases{0&in the area prediction mode\cr
             10\us m^2&in the point-to-point mode\cr}$$
   and where $\Delta h(s)$ is the function defined in (3.9) above.  Next we
   define a ``clutter factor''
   $$A_{f\!o}=\min\bigl[15,5\log\bigl(1+\alpha kh_{g1}h_{g2}\sigma_h(
      d_{Ls})\bigr)\bigr] \qquad\hbox{with $\alpha=4.77\cdot10^{-4}
        \us m^{-2}$}\eqno(4.10)$$
   and with $\sigma_h(s)$ defined in (3.10) above.

   Then $$A_{\rm diff}(s)=(1-w)A_k+wA_r+A_{f\!o}\eqno(4.11)$$ where the
   ``double knife edge attenuation'' $A_k$ and the ``rounded earth
   attenuation'' $A_r$ are yet to be defined.
   Set
   $$\theta=\theta_e+s\gamma_e\eqno(4.12)$$
   $$v_j={\theta \over 2}\left({k \over \pi} {d_{Lj}(s-d_L) \over
         s-d_L+d_{Lj}}\right)^{1/2},\qquad j=1,2\eqno(4.13)$$
   and then $$A_k={\rm Fn}(v_1)+{\rm Fn}(v_2)\eqno(4.14)$$ where Fn$(v)$
   is the Fresnel integral defined below.

   For the rounded earth attenuation we use a ``three radii'' method
   applied to Vogler's formulation of the solution to the smooth,
   spherical earth problem.  We set
   $$\gamma_0=\theta/(s-d_L)\qquad\gamma_j=2h_{ej}/d^2_{Lj},\quad
   j=1,2\eqno(4.15)$$
   $$\eqalignno{\alpha_j&=(k/\gamma_j)^{1/3},\qquad j=0,1,2&(4.16)\cr
           K_j&={1 \over i\alpha_jZ_g},\qquad j=0,1,2.&(4.17)\cr}$$
   Note that the $K_j$ are complex numbers.  To continue, we set
   $$\eqalignno{x_j&=AB(K_j)\alpha_j\gamma_j d_{Lj},
         \qquad j=1.2&(4.18)\cr
         x_0&=AB(K_0)\alpha_0\theta+x_1+x_2&(4.19)\cr}$$
   and then $$A_r=G(x_0)-F(x_1,K_1)-F(x_2,K_2)-C_1(K_0)\eqno(4.20)$$
   where $A=151.03$ is a dimensionless constant and the functions $B(K)$,
   $G(x)$, $F(x,K)$, and $C_1(K)$ are those defined by Vogler.

   In (4.14) and (4.20) we have finished the definition of $A_{\rm
   diff}$.  We should like, however, to complete the subject by defining
   more precisely the more or less standard functions mentioned above.
   The Fresnel integral, for example, may be written as
   $${\rm Fn}(v)= 20\,\log\biggl|{1 \over \sqrt{2i}}\int_v^\infty
      e^{i\pi u^2/2}du \biggr|.\eqno(4.21)$$

   For Vogler's formulation to the solution to the spherical earth
   problem, we first introduce the special Airy function
   $$\eqalign{{\rm Wi}(z)=&{\rm Ai}(z)+i{\rm Bi}(z)\cr
                         =&2{\rm Ai}(e^{2\pi i/3}z)\cr}$$
   where ${\rm Ai}(z)$ and ${\rm Bi}(z)$ are the two standard Airy
   functions defined in many texts.  They are analytic in the entire
   complex plane and are particular solutions to the differential
   equation $$w''(z)-zw(z)=0.$$
   First, to define the function $B(K)$ we find the smallest solution to
   the modal equation $${\rm Wi}(t_0)=2^{1/3}K{\rm Wi}'(t_0)$$ and then
   $$B=2^{-1/3}{\rm Im}\{t_0\}.\eqno(4.22)$$
   Finally, we also have
   $$\eqalignno{G(x)&=20\,\log(x^{-1/2}e^{x/A})&(4.23)\cr
                F(x,K)&=20\,\log\bigl|(\pi/(2^{1/3}AB))^{1/2} {\rm Wi}
        \bigl(t_0-(x/(2^{1/3}AB))^2\bigr)\bigr|&(4.24)\cr
                C_1(K)&=20\,\log\bigl|{1\over2}(\pi/(2^{1/3}AB))^{1/2}
        (2^{2/3}K^2t_0-1){\rm Wi}'(t_0)^2\bigr|&(4.25)\cr}$$
   where $A$ is again the constant defined above.

   It is of interest to note that for large x we find $F(x,K)\sim G(x)$,
   and that for those values of $K$ in which we are interested it is a
   good approximation to say $C_1(K)=20$\ dB.
   \newsec 4.2. Coefficients for the line-of-sight range.

   We begin by setting
   $$\eqalignno{d_2=&d_{Ls}&(4.26)\cr
                A_2=&A_{ed}+m_d d_2.&(4.27)\cr}$$

   Then there are two general cases.  First, if $A_{ed} \ge 0$
   $$\eqalignno{d_0=&\min\bigl({1\over 2}d_L,1.908\,kh_{e1}h_{e2}\bigr)
         &(4.28)\cr
                d_1=&{3\over 4}d_0+{1\over 4}d_L&(4.29)\cr
                A_0=&A_{\rm los}(d_0)&(4.30)\cr
                A_1=&A_{\rm los}(d_1)&(4.31)\cr}$$
   where the function $A_{\rm los}(s)$ is defined below.  The idea, now,
   is to devise a curve of the form
   $$A_{el}+K_1d+K_2\ln(d/d_{Ls})$$
   that passes through the three values $A_0$, $A_1$, $A_2$ at $d_0$,
   $d_1$, $d_2$, respectively,.  In doing this, however, we require
   $K_1,\ K_2 \ge 0$, and sometimes this forces us to abandon one or both
   of the values $A_0$, $A_1$.  We first define
   $$\eqalignno{K_2'=&\max\left[0,{(d_2-d_0)(A_1-A_0)-(d_1-d_0)(A_2-A_0)
   \over (d_2-d_0)\ln(d_1/d_0)-(d_1-d_0)\ln(d_2/d_0)}\right]&(4.32)\cr
     K_1'=&\bigl(A_2-A_0-K_2'\ln(d_2/d_0)\bigr)/(d_2-d_0)&(4.33)\cr}$$
   which, except for the possibility that the first calculation for
   $K_2'$ results in a negative value, is simply the straightforward
   solution for the two corresponding coefficients.  If $K_1'\ge 0$ we
   then have $$K_1=K_1',\qquad K_2=K_2'.\eqno(4.34)$$
   If, however, $K_1'< 0$, we define
   $$K_2''=(A_2-A_0)/\ln(d_2/d_0),\eqno(4.35)$$
   and if now $K_2''\ge 0$ then $$K_1=0,\qquad K_2=K_2''.\eqno(4.36)$$
   Otherwise, we abandon both $A_0$ and $A_1$ and set
   $$K_1=m_d,\qquad K_2=0. \eqno(4.37)$$

   In the second general case we have $A_{ed} < 0$.  We then set
   $$\eqalignno{d_0=&1.908\,kh_{e1}h_{e2}&(4.38)\cr
                d_1=&\max(-A_{ed}/m_d,d_L/4).&(4.39)\cr}$$
   If $d_0 < d_1$ we again evaluate $A_0$, $A_1$, and $K_2'$ as before.
   If $K_2' > 0$ we also evaluate $K_1'$ and proceed exactly as before.
   If, however, we have either $d_0 \ge d_1$ or $K_2'=0$, we evaluate
   $A_1$ and define $$K_1''=(A_2-A_1)/(d_2-d_1).\eqno(4.40)$$
   If now $K_1'' > 0$ we set $$K_1=K_1'',\qquad K_2=0;\eqno(4.41)$$ and
   otherwise we use (4.37).

   At this point we will have defined the coefficients $K_1$ and $K_2$.
   We finally set $$A_{el}=A_2-K_1d_2.\eqno(4.42)$$
   \newsec 4.2.1. The function $A_{\rm los}(s)$

   First we define the weighting factor
   $$w=1/\bigl(1+D_1k\Delta h/\max(D_2,d_{Ls})\bigr)\qquad{\rm with}\
           D_1=47.7\us m,\ D_2=10\us{km}.\eqno(4.43)$$
   Then $$A_{\rm los}=(1-w)A_d+wA_t\eqno(4.44)$$ where the ``extended
   diffraction attenuation'' $A_d$ and the ``two-ray attenuation'' $A_t$
   are yet to be defined.

   First, the extended diffraction attenuation is given very simply by
   $$A_d=A_{ed}+m_ds.\eqno(4.45)$$

   For the two-ray attenuation, we set
   $$\sin\psi={h_{e1}+h_{e2}\over\sqrt{s^2+(h_{e1}+h_{e2})^2}}
             \eqno(4.46)$$
   and $$R_e'={\sin\psi-Z_g\over\sin\psi+Z_g}\exp[-k\sigma_h(s)\sin\psi]
             \eqno(4.47)$$
   where $\sigma_h(s)$ is the function defined in (3.10) above.  Note
   that $R_e'$ is complex since it uses the complex surface transfer
   impedance $Z_g$.  Then
   $$R_e=\cases{R_e'&if $|R_e'|\ge\max(1/2,\sqrt{\sin\psi})$\cr
          {(R_e'/|R_e'|)}\sqrt{\sin\psi}&otherwise\cr}\eqno(4.48)$$
   We also set
   $$\delta'=2kh_{e1}h_{e2}/s\eqno(4.49)$$
   and
   $$\delta=\cases{\delta'&if $\delta'\le\pi/2$\cr
              \pi-(\pi/2)^2/\delta'&otherwise\cr}\eqno(4.50)$$
   Then finally $$A_t=-20\,\log|1+R_ee^{i\delta}|.\eqno(4.51)$$

   \newsec 4.3. Coefficients for the scatter range.

   Set
   $$\eqalignno{d_5=&d_L+D_s&(4.52)\cr
         d_6=&d_5+D_s\qquad{\rm with}\ D_s=200\us{km}.&(4.53)\cr}$$
   Then define
   $$\eqalignno{A_5=&A_{\rm scat}(d_5)&(4.54)\cr
                A_6=&A_{\rm scat}(d_6),&(4.55)\cr}$$
   where $A_{\rm scat}(s)$ is defined below.  There are, however, some
   sets of parameters for which $A_{\rm scat}$ is not defined, and it may
   happen that either or both $A_5$, $A_6$ is undefined.  If this is so,
   one merely sets $$d_x=+\infty\eqno(4.56)$$ and one can let $A_{es}$,
   $m_s$ remain undefined. In the more normal situation one has
   $$\eqalignno{m_s=&(A_6-A_5)/D_s&(4.57)\cr
         d_x=&\max\bigl[d_{Ls},d_L+X_{ae}\log(kH_s),
              (A_5-A_{ed}-m_sd_5)/(m_d-m_s)\bigr]&(4.58)\cr
         A_{es}=&A_{ed}+(m_d-m_s)d_x&(4.59)\cr}$$
   where $D_s$ is the distance given above, where $X_{ae}$ has been
   defined in (4.2), and where $H_s=47.7\us m$.
   \newsec 4.3.1. The function $A_{\rm scat}$.

   Computation of this function uses an abbreviated version of the
   methods described in Section~9 and Annex~III.5 of NBS~TN101.  First,
   set
   $$\eqalignno{\theta=&\theta_e+\gamma_es&(4.60)\cr
            \theta'=&\theta_{e1}+\theta_{e2}+\gamma_es&(4.61)\cr
                r_j=&2\,k\theta' h_{ej},\qquad j=1, 2.&(4.62)\cr}$$
   If both $r_1$ and $r_2$ are less than 0.2 the function $A_{\rm scat}$
   is not defined (or is infinite).  Otherwise we put
   $$ A_{\rm scat}(s)=10\,\log(kH\theta^4)
            +F(\theta s,N_s)+H_0 \eqno(4.63)$$
   where $F(\theta s,N_s)$ is the function shown in Figure~9.1 of TN101,
   $H_0$ is the ``frequency gain function'', and $H=47.7\us m$.

   The frequency gain function $H_0$ is a function of $r_1$, $r_2$, the
   scatter efficiency factor $\eta_s$, and the ``asymmetry factor'' which
   we shall here call $s_s$.  A difficulty with the present model is that
   there is not sufficient geometric data in the input variables to
   determine where the crossover  point is.  This is resolved by assuming
   it to be midway between the two horizons.  The asymmetry factor, for
   example, is found by first defining the distance between horizons
   $$d_s=s-d_{L1}-d_{L2}\eqno(4.64)$$
   whereupon
   $$s_s={d_{L2}+d_s/2 \over d_{L1}+d_s/2}.\eqno(4.65)$$
   There then follows that the height of the crossover point is
   $$z_0={s_sd\theta'\over(1+s_s)^2}\eqno(4.66)$$
   and then
   $$\eta_s={z_0\over Z_0}\big[1+(0.031-N_s 2.32\cdot10^{-3}
      +N_s^2 5.67\cdot10^{-6})e^{-(z_0/Z_1)^6}\big]\eqno(4.67) $$
   where
   $$ Z_0= 1.756\us{km}\qquad Z_1=8.0\us{km} $$
   The computation of $H_0$ then proceeds according to the rules in
   Section~9.3 and Figure~9.3 of TN101.

   The model requires these results at the two distances $s=d_5$,~$d_6$,
   described above.  One further precaution is taken to prevent anomalous
   results.  If, at $d_5$, calculations show that $H_0$ will exceed
   15~dB, they are replaced by the value it has at $d_6$.  This helps
   keep the scatter-mode slope within reasonable bounds.
   \fontdimen16\tensy=2.7pt \fontdimen17\tensy=2.7pt
    
   \newsec 5. Variability---the quantiles of attenuation.

   We want now to compute the quantiles $A(q_T,q_L,q_S)$ where $q_T$,
   $q_L$, $q_S$, are the desired fractions of time, locations, and
   situations, respectively.  In the point-to-point mode, we would want a
   two-fold quantile $A(q_T,q_S)$, but in the present model this is done
   simply by computing the three-fold quantile with $q_L$ equal to 0.5.

   Because the distributions involved are all normal, or nearly normal,
   it simplifies the calculations to rescale the desired fractions and to
   express them in terms of ``standard normal deviates.''  We use the
   complementary normal distribution
   $$Q(z)={1 \over \sqrt{2\pi}}\int_z^\infty e^{-t^2/2}dt$$
   and then the deviate is simply the inverse function
   $$z(q)=Q^{-1}(q).$$
   Thus if the random variable $x$ is normally distributed with mean
   $X_0$ and standard deviation $\sigma$, its quantiles are given by
   $$X(q)=X_0+\sigma z(q).$$

   Setting $$z_T=z(q_T),\quad z_L=z(q_L),\quad z_S=z(q_S),$$ we now ask
   for the quantiles $A(z_T,z_L,z_S)$.  In these rescaled variables, it
   is as though all probabilities are to be plotted on normal probability
   paper.  In the case of the point-to-point mode we will simply suppose
   $z_L=0$.

   First we define
   $$A'=A_{\rm ref}-V_{\rm med}-Y_T-Y_L-Y_S\eqno(5.1)$$
   where $A_{\rm ref}$ is the reference attenuation defined in Section~4,
   and where the adjustment $V_{\rm med}$ and the deviations $Y_T$,
   $Y_L$, $Y_S$, are defined below.  The values of $Y_T$ and $Y_L$ depend
   on the single variables $z_T$ and $z_L$, respectively.  The value of
   $Y_S$, on the other hand, depends on all three standard normal
   deviates.

   The final quantile is a modification of $A'$ given by
   $$A(z_T,z_L,z_S) = \cases{A'&if $A'\ge 0$\cr \displaystyle A'
     {29-A'\over 29-10A'}&otherwise.\cr}\eqno(5.2)$$

   An important quantity used below is the ``effective distance.''  We
   set $$d_{ex}=\sqrt{2a_1h_{e1}}+\sqrt{2a_1h_{e2}}+a_1(kD_1)^{-1/3}
      \eqno(5.3)$$ with
   $$a_1=9000\us{km},\qquad D_1=1266\us{km}.$$
   Then the effective distance is given by
   $$d_e=\cases{D_0d/d_{ex}&for $d \le d_{ex}$\cr
            D_0+d-d_{ex}&for $d \ge d_{ex}$\cr}\eqno(5.4)$$
   with $D_0=130\us{km}$.
   \newsec 5.1.  Time variability.

   Quantiles of time variability are computed using a variation of the
   methods described in Section~10 and Annex~III.7 of NBS~TN101, and also
   in CCIR Report~{238-3}.  Those methods speak of eight or nine discrete
   radio climates, of which seven have been documented with corresponding
   empirical curves.  It is these empirical curves to which we refer
   below.  They are all curves of quantiles of deviations versus the
   effective distance $d_e$.

   The adjustment from the reference attenuation to the all-year median
   is$$V_{\rm med}=V_{\rm med}(d_e,{\rm clim})\eqno(5.5)$$where the
   function is described in Figure~10.13 of TN101.

   The deviation $Y_T$ is piecewise linear in $z_T,$ and may be written in
   the form
   $$Y_T=\cases{\sigma_{T-}z_T&$z_T \le 0$\cr
         \sigma_{T+}z_T&$0\le z_T \le z_D$\cr
         \sigma_{T+}z_D+\sigma_{TD}(z_T-z_D)&$z_D \le z_T$\cr}\eqno(5.6)$$

   The slopes (or ``pseudo-standard deviations'')
   $$\eqalign{\sigma_{T-}=&\sigma_{T-}(d_e,{\rm clim})\cr
              \sigma_{T+}=&\sigma_{T+}(d_e,{\rm clim})\cr}\eqno(5.7)$$
   are obtained from TN101 in the following way.  For $\sigma_{T-}$ we
   use the .90 quantile and divide the corresponding ordinates by $z(.90)
   = -1.282$.  For $\sigma_{T+}$ we use the .10 quantile and divide by
   $z(.10) = 1.282$.

   The remaining constants in (5.6) pertain to the ``ducting,'' or low
   probability, case.  We write
   $$z_D=z_D({\rm clim}),\qquad \sigma_{TD}=C_D({\rm clim})\sigma_{T+}
                \eqno(5.8)$$
   where values of $z_D$ and $C_D$ are given in Table~5.1.  In that table
   we have also listed values of $q_D = Q(z_D)$.

      \midinsert
   \centerline{Table 5.1.  Ducting (low probability) constants}
   \medskip
     \hbox to \hsize{\hfil
     \vbox{\tabskip=0pt\offinterlineskip
     \def\trule{\noalign{\hrule}}
   \halign to 30em{\strut#& \vrule#\tabskip=1em plus 2em&
      #\hfil& \vrule#& \quad#\hfil&
         \vrule#& \hfil#& \vrule#& \hfil#& \vrule#
         \tabskip=0pt\cr\trule
   &&\vrule height3ex width0ex depth1.8ex
         \qquad Climate&&$q_D$&&$z_D$&&$C_D$&\cr\trule
   &&Equatorial&&.10&&1.282&&1.224&\cr\trule
   &&Continental Subtropical&&\llap{$\approx$}.015&&2.161&&.801&\cr\trule
   &&Maritime Subtropical&&.10&&1.282&&1.380&\cr\trule
   &&Desert&&\hfil 0&&$\infty$\hfil&&$-$\hfil&\cr\trule
   &&Continental Temperate&&.10&&1.282&&1.224&\cr\trule
   &&Maritime Temperate Overland&&.10&&1.282&&1.518&\cr\trule&&Maritime
   Temperate Oversea&&.10&&1.282&&1.518&\cr\trule}}
     \hfil}
      \endinsert
   \newsec 5.2. Location variability.

   We set
   $$Y_L=\sigma_Lz_L\eqno(5.9)$$
   where$$\sigma_L=10k\Delta h(d)/(k\Delta h(d)+13)$$
   and $\Delta h(s)$ is defined in (3.9) above.

   \newsec 5.3. Situation variability.

   Set $$\sigma_S=5+3e^{-d_e/D}\eqno(5.10)$$ where $D=100\us{km}$.  Then
   $$Y_S=\left(\sigma_S^2+{Y_T^2\over 7.8+z_S^2}+{Y_L^2\over 24+z_S^2}
            \right)^{1/2}z_S \eqno(5.11)$$
   This latter is intended to reveal how the uncertainties become greater
   in the wings of the distributions.
   \newsec 6.  Addenda---numerical approximations.

   Part of the algorithm for the ITM consists in approximations for the
   standard functions that have been used.  In these approximations,
   computational simplicity has often taken greater priority than accuracy.

   The Fresnel integral is used in \S 4.1.1 and is defined in (4.21).  We
   have (for $v>0$)
   $$ {\rm Fn}(v)\approx \cases{6.02+9.11v-1.27v^2&if $v\le 2.40$\cr
                    12.953+20\log\,v&otherwise\cr}\eqno(6.1) $$

   The functions $B(K)$, $G(x)$, $F(x,K)$, $C_1(K)$, which are used in
   diffraction around a smooth earth, are also used in \S 4.1.1 and are
   defined in (4.22)--(4.25).  We have
   $$\eqalignno{B(K)\approx& 1.607-|K| &(6.2)\cr
      G(x)=&.05751x-10\log\,x &(6.3)\cr
      F(x,K)\approx&\cases{F_2(x,K)&if $0<x\le 200$\cr
               G(x)+0.0134xe^{-x/200}(F_1(x)-G(x))&if $200<x<2000$\cr
               G(x)&if $2000\le x$\cr}&(6.4)} $$
   where
   $$ F_1(x)=40\log(\max(x,1))-117 \eqno(6.5) $$
   $$ F_2(x,K)=\cases{F_1(x)&
            \hskip -5ex if $|K|<10^{-5}$ or $x(-\log|K|)^3>450$\cr
          2.5\cdot10^{-5}x^2/|K|+20\log|K|-15&otherwise\cr}\eqno(6.6) $$
   The final approximation here is
   $$ C_1(K)\approx 20 \eqno(6.7) $$

   To complete this section we have the two functions, $F(\theta d)$ and
   $H_0$, used for tropospheric scatter.  First,
   $$ F(D,N_s)=F_0(D)-0.1(N_s-301)e^{-D/D_0} \eqno(6.8) $$
   where
   $$ D_0=40\us{km} $$
   and (when $D$ is given in meters)
   $$ F_0(D)=\cases{
       133.4+0.332\cdot10^{-3}D-10\,\log D &for $0<D\le10\us{km}$\cr
       104.6+0.212\cdot10^{-3}D-2.5\,\log D &for $10<D\le70\us{km}$\cr
        71.8+0.157\cdot10^{-3}D+5\,\log D &otherwise\cr}\eqno(6.9) $$

   The frequency gain function may be written as
   $$ H_0=H_{00}(r_1,r_2,\eta_s)+\Delta H_0 \eqno(6.10) $$
   where
   $$ \Delta H_0=6(0.6-\log\eta_s)\log s_s \log r_2/s_sr_1 \eqno(6.11) $$
   and where $H_{00}$ is obtained by linear interpolation between its
   values when $\eta_s$ is an integer.  For $\eta_s=1,\ldots,5$ we set
   $$ H_{00}(r_1,r_2,j)=
          {1\over2}[H_{01}(r_1,j)+H_{01}(r_2,j)] \eqno(6.12) $$
   with
   $$ H_{01}(r,j)=\cases{
         10\,\log(1+24r^{-2}+25r^{-4})&$j=1$\cr
         10\,\log(1+45r^{-2}+80r^{-4})&$j=2$\cr
         10\,\log(1+68r^{-2}+177r^{-4})&$j=3$\cr
         10\,\log(1+80r^{-2}+395r^{-4})&$j=4$\cr
         10\,\log(1+105r^{-2}+705r^{-4})&$j=5$\cr} \eqno(6.13) $$
   For $\eta_s>5$ we use the value for $\eta_s=5$, and for $\eta_s=0$ we
   suppose
   $$ H_{00}(r_1,r_2,0)=10\,\log\left[\left(1+{\sqrt2\over r_1}\right)^2
         \left(1+{\sqrt2\over r_2}\right)^2
        {r_1+r_2\over r_1+r_2+2\sqrt2}\right] \eqno(6.14) $$
   In all of this, we truncate the values of $s_s$ and $q=r_2/s_sr_1$ at
   0.1 and 10.
    
   \bye
