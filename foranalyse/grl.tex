\newpage
\section{Grønland}
I denne afsnit vil Grønland blive beskrevet. Her vil størrelsen af landet, byerne samt bygderne og befolkningstallet blive analyseret.

\subsection{Fakta om Grønland}
\begin{wrapfigure}{l}{0.5\textwidth}
	\includegraphics[width=0.5\textwidth]{Figure/greenland.jpg}
	\caption{\textit{Afstandene mellem byerne er store og største delen af Grønland er dækket med is}}
	\label{fig:greenland}
\end{wrapfigure} 
Grønland er verdens største (ikke kontinental) ø \ref{fig:greenland}, som ligger imellem den nordlige Atlantiske Ocean og den Arktiske Hav. Størrelsen af landet er $2.166.086 km^{2}$, hvor $410.449 km^{2}$ er isfrit og kystlinjen strækker sig over 44.000 km. Den isfrie del af Grønland er bjergrig med mange fjorde\cite{grlStat}. \\
Grønland ligger i arktis, det betyder at temperaturen i vinter perioden ligger omkring $-20\deg C$ i den nordligste by og $-3\deg C$ i den sydligste by, hvor imod i sommer perioden ligger temperaturen omkring $7\deg C$ i den nordligste ny og $8\deg C$ i den sydligste by\cite{qaanaaq}\cite{nanortalik}.\\
Den 1. januar 2016 boede der 55.847 personer og grønlands statistik forventer at befolkningsantallet vil falde til 53.000 personer i de kommende 24 år\cite{grlStatWeb}.\\\\
Hovedstaden hedder Nuuk og ligger på grønlands vestkyst, hvor i dag bor omkring 12.800 personer.\\
På tabel \ref{tab:befolkning}, kan befolkningsantallet ses delt i kommunerne.
\newpage
\begin{table}[h]
\centering
    \begin{tabular}{|l|l|}
    \hline
    2016                     & ~      \\ \hline
    Kommune Kujalleq         & 6.811  \\ \hline
    Kommuneqarfik Sermersooq & 22.480 \\ \hline
    Qeqqata Kommunia         & 9.423  \\ \hline
    Qaasuitsup Kommunia      & 17.008 \\ \hline
    Udenfor kommunerne       & 125    \\ \hline
    \end{tabular}
    \caption{Befolkningsantal fordelt i kommuner}
    \label{tab:befolkning}
\end{table}

\subsection{Infrastruktur af telekommunikation}
Som tidligere nævn, er Grønland et stort land, hvor byerne ikke er forbundet med veje, da afstanden mellem byerne og bygderne er stor. Men Tele Greenland har et netværk der forbinder alle byer og bygder i Grønland til omverdenen. Tele Greenland har i dag delt telekommunikationsinfrastrukturen i tre zoner, se figur \ref{fig:zoner}.\\
\begin{figure}[h]
	\centering
	\includegraphics[width=0.8\textwidth]{figure/zoner.PNG}
	\caption{Telekommunikationsinfrastruktur delt i 3 zoner}
	\label{fig:zoner}
\end{figure} 
Den røde linje er søkablet, som er forbundet til to byer i Grønland (Nuuk og Qaqortoq) fra Island og Canada.\\
Den blå cirkel er zonen, hvor byerne og bygderne er forbundet med radiokæde. Cirklen strækker sig fra Nanortalik (Sydgrønland) til Uummannaq (Nordgrønland).\\
Den zone er de grønne cirkler, som er placeret i den nordlige del af Grønland og øst Grønland. byerne og bygderne i de grønne zoner er forbundet til satellitter.\\ 
Den overordnede telekommunikationsinfrastruktur kan ses i bilag \ref{bilag:telesites}\\
De grønne zoner har som sagt kommunikation via satellitter, og derfor er kommunikations hastigheden også ringere end de to andre zoner.

\subsection{Kapacitet}


\subsection{Services}
I de seneste år, viser tallene, at både den mobile bredbånd- og mobiltelefonabonnenter er stigende, hvorimod abonnenter for bredbånd via fastnet og telefonlinjer er faldende.

\subsection{Dataforbrug}

\subsection{Busy hour}
Busy hour er en periode som navnet siger, en periode på 1 time hvor data trafikken er på sin højeste inden for 24 timer. Ifølge den amerikanske multinationale firma Cisco har den globale nettrafik øget med 34 procent i 2014. Cisco forudsiger at den globale busy-hour trafik til øges med en faktor på 3,4 fra 2014 til 2019, dvs. den globale busy-hour trafik vil nå 1,7 Pbps (petabits per second) i år 2019\cite{busy-hour}.\\
Ud fra de oplysninger har Tele Greenland lavet et prognose over hvor meget befolkningen i Grønland kommer til at bruge i busy-hour perioden. I den næste underafsnit vil prognosen blive beskrevet og forklaret.

\subsection{Prognose}
I takt med at flere benytter sig af den mobile bredbånd- og mobiltelefonabonnenter, stiger også kapacitetsbehovet for brugerne og kommunikationsnetværket i Grønland.\\
Tele Greenland har udarbejdet en prognose over kapacitetsbehovet over tid (fra 2017 til 2022) og sted i Grønland.\\
\textit{Bemærk, at udregningerne er regnet ved at antage, at den grønlandske befolkning har det samme busy-hour trafik pr. individ.}.\\
Først regnes der baglæns, dvs. den globale busy-hour trafik i år 2014 findes først, for at beregne den globale busy-hour trafik i år 2017, ved brug af vækst raten fra 2014 til 2019.\\

\begin{align}
	Busy Hour_{2014}& = \frac{Busy Hour_{2019}}{vækstfaktor} \nonumber\\
	&= \frac{1,7 \cdot 10^{15}}{3,4} \nonumber\\
	&= 500 Tbps
\end{align}

Ud fra Busy-Hour trafikken fra 2014 kan Busy-Hour trafikken prognosen udregnes sammen med vækstraten.\\ Hvor vækstraten findes ved:
\begin{align}
	vækstrate &= vækstfaktor^\frac{1}{5} \nonumber\\
	&\simeq 1,28
\end{align}

Nu kan den globale Busy-Hour trafik for frem i tiden udregnes:

\begin{align}
	Busy Hour &= Busy Hour_{2014} \cdot vækstrate^n
\end{align}
\begin{tabbing}
Hvor:\\
	$ n $ = er 0 i 2014, 1 i 2015, 2 i 2016 osv.
\end{tabbing}
Fra 2017 til 2022 vil Busy-hour trafikken se således ud, se tabel\ref{tab:prognose}:
\begin{table}[!h]
	\centering
    \begin{tabular}{|c|c|}
    \hline
    År   & Busy-hour i Pbps \\ \hline
    2017 & 1,04             \\ \hline
    2018 & 1,33             \\ \hline
    2019 & 1,7              \\ \hline
    2020 & 2,17             \\ \hline
    2021 & 2,77             \\ \hline
    2022 & 3,54             \\ \hline
    \end{tabular}
    \label{tab:prognose}
    \caption{Busy-hour trafik prognose fra 2017 til 2022}
\end{table}
\begin{figure}[h]
	\centering
	\includegraphics[width=0.8\textwidth]{figure/busyHourPrognose.pdf}
	\caption{Busy-hour trafik prognose fra 2017 til 2022}
	\label{fig:prognose}
\end{figure}
Nu hvor Busy-hour trafik prognosen er beregnet, kan Busy-hour trafik pr. individ beregnes ud fra prognosen for den globale befolkningstal. Dermed kan prognosen for Busy-hour trafik i grønland beregnes.\\
Prognosen for den globale befolkningstal kan ses i tabel \ref{tab:globalBefolkning}
\begin{table}
	\centering
    \begin{tabular}{|c|c|c|}
    \hline
    År   & Globalt befolkningstal & Busy-hour trafik pr. individ i kpbs \\ \hline
    2017 & 7515284,153            & 139                          \\ \hline
    2018 & 7597175,534            & 175                          \\ \hline
    2019 & 7678174,656            & 221                          \\ \hline
    2020 & 7758156,792            & 280                          \\ \hline
    2021 & 7837028,569            & 354                          \\ \hline
    2022 & 7914763,610            & 448                          \\ \hline
    \end{tabular}
    \caption{Prognose for globalt befolkningstal}
    \label{tab:globalBefolkning}
\end{table}
Tal fra Grønlandsk statistik, som viser befolkningstallet for hele Grønland de seneste 5 år (2011 til 2016) kan ses i tabel \ref{tab:grlBefolkning}, her kan prognoses også ses. Tallene er beregnet ud fra gennemsnitstallet gemmen de seneste 5 år, og addere tallet hver år der går, frem til 2022.
\begin{table}
	\centering
    \begin{tabular}{|l|l|l|l|l|l|l|l|}
    \hline
    År        & 2012   & 2013   & 2014   & 2015   & 2016   & ~      & $\Delta$ \\ \hline
    Statistik & 56.749 & 56.370 & 56.282 & 55.984 & 55.846 & ~      & -154                \\ \hline
    År        & 2017   & 2018   & 2019   & 2020   & 2021   & 2022   & ~                   \\ \hline
    Prognose  & 55.692 & 55.538 & 55.385 & 55.231 & 55.077 & 54.923 & ~                   \\ \hline
    \end{tabular}
    \caption{Befolkningstal de seneste 5 år, og prognose frem til 2022}
    \label{tab:grlBefolkning}
\end{table}
Prognosen for den globale Busy-hour trafik pr. individ og prognoses for befolkningstallet i Grønland frem til 2022 er beregnet. Nu kan prognosen for kapacitetsbehovet i Grønland frem til 2022 findes, det findes ved at multiplicere de to tal. Prognosen kan ses i tabel \ref{tab:kapacitetPrognose}.
\begin{table}
	\centering
    \begin{tabular}{|c|c|c|c|c|c|c|}
    \hline
    År                 & 2017      & 2018      & 2019       & 2020       & 2021       & 2022       \\ \hline
    Kapacitetsprognose & 7.721.583 & 9.729.586 & 12.262.526 & 15.458.505 & 19.492.132 & 24.584.091 \\ \hline
    \end{tabular}
    \caption{Kapacitetsbehov frem til 2022 i Grønland}
    \label{tab:kapacitetPrognose}
\end{table}
\begin{figure}
	\centering
	\includegraphics[width=1\textwidth]{figure/kapacitetsbehov.pdf}
	\caption{Kapacitetsbehov frem til 2022 i Grønland}
	\label{fig:kapacitetsbehov}
\end{figure}